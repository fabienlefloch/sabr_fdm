\documentclass[]{rAMF2e}
%\usepackage[square]{natbib}
%\usepackage{authblk}
\usepackage{listings}
\usepackage[center]{caption}


\begin{document}
\doi{}
\issn{}  \issnp{}
\jvol{00} \jnum{00} \jyear{2013} %\jmonth{January--March}
\def\jobtag{}
\publisher{Unpublished}
\jname{}

\markboth{Fabien {Le Floc'h}, Gary Kennedy}{Draft}

\title{SABR Characteristic Function}
\author{Fabien {Le Floc'h}$^\star$\thanks{{\em{Correspondence Address}}: Calypso Technology, 106 rue de La Bo\'{e}tie, 75008 Paris. Email: \texttt{fabien\_lefloch@calypso.com} \vspace{6pt}} and Gary Kennedy$^\dag$}
\affil{$^\star$Calypso Technology, 106 rue de La Bo\'{e}tie, 75008 Paris\\$^\dag$Clarus Financial Technology, London}
%
\date{\today}
\received{v1.0 released September 2013}

\maketitle
\newcommand{\sgn}{\mathop{\mathrm{sgn}}}

\section{Introduction}
We write SABR as:
\begin{align}
dF &= \epsilon \alpha F^{\beta} dW_1 \\
d\alpha &= \epsilon \nu \alpha dW_2 
\end{align}
Feynman Kac on an expansion of $\mathbb{E}(e^{i\Phi x})$ of SABR leads to
\begin{align}
\frac{\partial y}{\partial t} + \frac{1}{2}\epsilon^2\nu^2\alpha^2 \frac{\partial^2 y}{\partial  \alpha^2} + s_1 \alpha^2 y &= 0
\end{align}
where $s_1 = O(\epsilon^2)$ and $y(T,T) = e^{s_2 \alpha + s_3 \alpha^2}$

\end{document}
